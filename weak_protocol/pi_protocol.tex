\documentclass[11pt]{article}
\usepackage{amsmath, amssymb, amscd, amsthm, amsfonts}
\usepackage{graphicx}
\usepackage{hyperref}
\usepackage{authblk}
\usepackage{datetime}

\oddsidemargin 0pt
\evensidemargin 0pt
\marginparwidth 40pt
\marginparsep 10pt
\topmargin -20pt
\headsep 10pt
\textheight 8.7in
\textwidth 6.65in
\linespread{1.2}

\title{Weak PRF Protocol: Pseudocode}
\author{Vivek Sharma}
\date{Jan 11, 2020}

\newcommand{\rr}{\mathbb{R}}

\newcommand{\al}{\alpha}
\DeclareMathOperator{\conv}{conv}
\DeclareMathOperator{\aff}{aff}

\begin{document}

\maketitle


\section{Fully Distributive Evaluation Protocol}
The key is structured as a vector in $\mathbb{Z}_2$


The protocol is divided into three phases:

\subsection{Phase 1:} 
Each server $S_{i}$ holds replicated additive shares of key $k_{i} \in \mathbb{Z}_{n}^{2}$ and $x_{i} \in \mathbb{Z}_{n}^{2}$ and computes h, which is the multiplication of key and input over $\mathbb{Z}_2$. This computation is performed locally. 
\subsection{Phase 2:}

\begin{enumerate}
	\item Each server, at this point have locally computed their shares, which was the multiplication of two vectors.
	\item Server 1 randomly chooses a value $c \in \mathbb{Z}_{3}^{m}$ and each bit of value is converted to it's 2-bit representation to form $c_{0} and c_{1}$ respectively. 
	\item Meanwhile, Server 1, 2 and 3 runs sub-protocol for m instances(m is the length of additive share and also the value of c, which is with server 1.
	\subitem For $1 \leq j \leq m$:
	\subitem Each server $s_{i} , i \in {1,2,3}$ share their input $h_{i,j}$ [\,Note: $h_{i,j}$ is the input of server $s_{i}$ in $j^{th}$ iteration ]\,
	\subitem Compute combined XOR of their input: $comb := h_{1} \oplus h_{2} \oplus h_{13}$
	\subitem Multiply one part of $c,(c_{0}),$ with comb and other part $(c_{1})$ with $\neg{comb}$ and XOR both the result, this forms $d_{0}$.
	\subitem To compute $d_{1}$, XOR the $c_{0}$ and $\neg{c_{1}}$, and multiply the result with the XOR of secret share of the servers.
	\subitem The final result $d = d_{0}, d_{1} \in \{ 0,1\}^{2}$ is converted back into $\mathbb{Z}_{3}$
	\subitem At the end of this phase, Server 1 has $c \in \{0,1\}^m and $ Server 2 has received the output $d \in \mathbb{Z}_{3}$
\end{enumerate}

\subsection{Phase 3:}


 
Explanation:

\end{document}
